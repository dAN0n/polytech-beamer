% Each slide goes in a frame environment.
% The first one should be your title page, which is
% set up automatically by Beamer.
\begin{frame}
  \titlepage
\end{frame}

% Some people like to include a table of contents.
% (I find them tacky; your audience shouldn't need a map!)
\begin{frame}
  \frametitle{Contents}
  \toc
\end{frame}

\section{Basics}
% Every frame should have a \frametitle
\begin{frame}
  \frametitle{Title of the first frame}
  This frame has very \textit{\textbf{little}} content.

  \begin{alertblock}{Caution}
    This example is meant to be used alongside its source!
    If you are only viewing the PDF, this document will not be useful to you.
  \end{alertblock}
  \note{Денис - дурак}
\end{frame}

\begin{frame}[fragile] % The argument "fragile" protects \verb, which is badly-behaved.
  \frametitle{Text formatting}
  Beamer supports the same text-formatting commands as ordinary \LaTeX{}.

  However, for best results, you should use \verb|\alert| instead of \verb|\emph|:

  \begin{itemize}
  \item \emph{emph text}
  \item \alert{alert text}
  \end{itemize}

  \textbf{Bold text} should be avoided; you can't be sure it will be distinct on a projector.
\end{frame}

\section{Blocks}
\begin{frame}
  \frametitle{Use of block environments I}
  Beamer supports the standard block environments.

  \begin{theorem}
    You can write theorems.
  \end{theorem}

  \begin{proof}
    They can have proofs!
  \end{proof}
\end{frame}

\begin{frame}
  \frametitle{Use of block environments II}
  \begin{example}
    You can also give examples.
  \end{example}

  \begin{definition}
    And define \alert{terms}.
  \end{definition}
\end{frame}

\begin{frame}
  \frametitle{Use of block environments III}
  \begin{block}{Custom block}
    You can even create custom blocks!
  \end{block}

  \begin{exampleblock}{Custom example}
    And custom example blocks!
  \end{exampleblock}
\end{frame}

\section{Math}
\begin{frame}
  \frametitle{Using math in Beamer}

  Math works both inline: $x^{2} + y^{2} = z^{2}$
  and display:
  \[ \int_{-\infty}^{\infty} e^{-x^{2}} \, \mathrm{d}x = \sqrt{\pi} \]
\end{frame}

\section{Lists}
\begin{frame}
  \frametitle{Lists in Beamer}
  Beamer also has support for ordered and unordered lists.

  \begin{enumerate}
  \item An ordered item.
  \begin{enumerate}
    \item 1
    \item 2
    \item 3
  \end{enumerate}
  \item Another ordered item.
    \begin{itemize}
    \item An unordered subitem.
    \item Another unordered subitem.
    \end{itemize}
  \item A third ordered item.
  \end{enumerate}
\end{frame}

\section{Overlays}
\begin{frame}
  \frametitle{Overlay specifications}
  Beamer allows you to set up animated slides using ``overlays''.
  Use this with caution---less is more!

  Overlays are created by specifying the numbers of the subslides on which the object should appear.
\end{frame}

\begin{frame}[fragile] % The "fragile" argument protects \verb, which is badly-behaved
  \frametitle{Example of overlay with itemize}
  The following items will appear one at a time.
  Each is created using the \verb|\item| call shown.
  \begin{itemize}
  \item<1-> \verb|\item<1->|
  \item<2-> \verb|\item<2->|
  \item<3-> \verb|\item<3->|
  \item<4-> \verb|\item<4->|
  \end{itemize}
\end{frame}

\subsection{Simplified syntax}
\begin{frame}[fragile]
  \frametitle{Example of simplified overlay with itemize}

  Beamer supports a simplified syntax for revealing lists one item at a time.
  \begin{block}{Code}
  \begin{verbatim}
\begin{itemize}[<+->]
  \item First item
  \item Second item
  \item Third item
\end{itemize}
  \end{verbatim}
  \end{block}

  \begin{itemize}[<+->]
    \item First item
    \item Second item
    \item Third item
  \end{itemize}
\end{frame}

\subsection{Simplified syntax2}
\begin{frame}[fragile]
  \frametitle{Example of simplified overlay with text}
  You can also use a simplified syntax to reveal a slide from top to bottom using \verb|\pause|.

  \pause

  This text will appear only on the second slide.
\end{frame}

\section{Structure and ToC}
\begin{frame}[fragile]
  \frametitle{Document structure}
  A Beamer presentation can be structured using \verb|\section| and \verb|\subsection|, just like a paper.

  These commands go \alert{between} the slides.
  They do not change the text on your slides, but they show up in the Table of Contents, and some themes show them in a sidebar or tree.
\end{frame}

\begin{frame}
  \frametitle{Tables of contents}
  Beamer supports automatically building tables of contents.

  \begin{alertblock}{Caution}
    Be careful how you use the table of contents.
    Just throwing an outline slide up and then using a minute to read it out is a waste of your audience's time.
  \end{alertblock}
\end{frame}

\section{Slide formatting}
\begin{frame}
  \frametitle{Multicolumn slides}
  Beamer supports slides with multiple columns.

  \begin{columns}
    \column{0.4\textwidth} %Using .4\textwidth leaves some gutter space between the columns, which is good with blocks.
    \begin{theorem}
      Here is a theorem.
    \end{theorem}

    \column{0.4\textwidth}
    \begin{proof}
      And here is its proof.
    \end{proof}
  \end{columns}
\end{frame}

\begin{frame}[fragile]
  \lstset{style=vhdl}
\begin{lstlisting}[label=lst:VHDL,caption=Описание схемы]
entity lab2 is
port(
  SW0,SW1,SW2,SW3,SW4 : in bit;
  LED0,LED1,LED2 : out bit;
  LED3,LED4,LED5 : out boolean
  );
end lab2;
architecture rtl of lab2 is
  signal TEMP : bit := '0';
begin
  LED2 <= '0';
  temp <= SW0 or SW1;
  LED1 <= TEMP and SW2;
  LED0 <= not TEMP;
  LED3 <= not(SW3 > SW4);
  LED4 <= not(SW3 = SW4);
  LED5 <= not(SW3 < SW4);
end rtl;
\end{lstlisting}

\end{frame}

\begin{frame}[fragile, allowframebreaks]{Матлаб}
  \lstset{style=matlab}
\begin{lstlisting}[label=lst:VHDL,caption=Описание схемы]
function [] = Main ()
clc;
clear all;                     
close all;
initialX = [3; 8]; % Начальная точка  
% initialX = [11; 4]; %%%%% Вторая начальная точка  
index = [-31,-34,4,286,388]; % Значения всех аргументов
e = 0.1;                    
H = [index(1)*2, index(3); index(3), index(2)*2];         
% Открытие файла вывода для записи результатов
fileID = fopen('results.txt', 'wt');
if (fileID == -1)
    error('Не удалось открыть файл вывода.');
    return;
end

% Функция построение графика метода
function [] = PlotGraph (v)
% Область построения
x_1=2:.1:6;
x_2=5:.1:9;
% x_1=4:.1:12; %%%%% Для второй начальной точки                 
% x_2=3:.1:9;
[x_1,x_2]=meshgrid(x_1,x_2);
w=(index(1)*x_1.^2 +index(2)*x_2.^2 + index(3)*x_1.*x_2 + index(4)*x_1 + index(5)*x_2 );

figure;
hold on;
contour(x_1,x_2,w,30);
plot(x, y, '.-k');
contour(x_1,x_2,w,v);
xlabel('x1');
ylabel('x2');
hold off;
end

% Функция построения графика сравнения кол-ва итераций
function [] = PlotIterCountGraph ()
figure;
surf(from:1:2*to,from:1:2*to,N);
xlabel('x1');
ylabel('x2');
zlabel('Кол-во итераций');
colorbar

figure;
contourf(from:1:2*to,from:1:2*to,N)
xlabel('x1');
ylabel('x2');
c = colorbar;
c.Label.String = 'Кол-во итераций';
end

% Вычисление функции и значение её производной
function [fX, dfX] = derivative(X) 
% Вычисление значения функции от Х
fX = index(1) * X(1)^2 + index(2) * X(2)^2 + index(3) * X(1) * X(2) + index(4) * X(1) + index(5) * X(2);
% Вычисление частных производных по Х1 и Х2 соответственно
dfX = [index(1)*2 * X(1) + index(3) * X(2) + index(4); index(2)*2 * X(2) + index(3) * X(1) + index(5)];
end
\end{lstlisting}

\end{frame}